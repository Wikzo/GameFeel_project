\section{Introduction}
Game design is a difficult discipline that typically requires many years of experience to get right. An important aspect of game design is \textit{game feel}, i.e., the sensation of control in a videogame, as described by Swink \cite{swink}. Game feel is the extension of the player's senses, as well as the pleasure of learning and mastering a skill. It's caused by a constant feedback loop between player and system. Game feel happens from the moment players form an intention in their heads and press a button, to when they see the results of the virtual character moving on-screen. Whenever players interact with a game, they are exposed to the feel of the game. This means that the feel can make or break the player experience.

The current state of designing game feel is that of an iterative process wherein designers have to carefully consider and tweak every aspect to get the game feeling ``right". This process often relies on the gut feeling of the game designers, typically combined with extensive testing. There is little practical framework or vocabulary that designers can build upon, whether they want to have a game that feels like \textit{Super Mario Bros.}\ or \textit{Sonic The Hedgehog}. Instead, designers will have to reinvent the wheel, often by means of trial and error.

Even though game feel is about virtual non-physical experiences, it can still be related to the physical world. Every object in the physical world has properties that define their unique feel, e.g., their textures, shapes and interactive properties. While a bowling ball feels massive and heavy, a knife feels sharp, pointy and thin. When it comes to games, designers and players don't have the required vocabulary to talk about these characteristics. One of the main problems when discussing game feel is the fact that it's an holistic experience consisting of many contributing elements, such as graphics, physical simulation, sounds, input devices, refresh rate of the screen, etc. It is difficult to isolate the impact of each of these aspects.

West argued that \textit{``the `feel' of a game is, in large part, described in terms of how responsive it is. Very often a game will be described as `laggy' or `sluggish', and by contrast other games will be `tight' or `fast'."} \cite{measure_lag} However, these are somewhat loose terms, and it is unclear when exactly a game goes from feeling `sluggish' to `tight'.

Therefore, this paper sets out to investigate two questions: A) Which words players use to describe the feel of a game, and B) What parameters yield those specific descriptions. The project is based on a simple 2D platforming game where players control a rolling ball with the keyboard. Inspired by Swink's discussion  about \textit{Response Metrics} \cite{swink}, the game changes two types of parameters between each round: how fast the ball accelerates and how fast it decelerates when moving horizontally. Hence, the velocity of the player is modulated over time, creating different types of game feel.

%This pattern is inspired by ADSR envelopes (Attack-Decay-Sustain-Release) \cite{adsr}, which are often used in order make an electronic musical instrument mimic the sound of a mechanical instrument, e.g., the sound of a pipe organ or a guitar string. This is achieved by modulating the amplitude over time.

To test the impact of these parameters, an experiment was conducted by uploading the game to the Internet where people could play it in their browsers. Players had to play the same game four times, where each round changed the acceleration and deceleration. Between the rounds, players  tried to describe the feel of the controls, as well as rate how strongly they felt this/these feeling(s). X number of people participated.

This paper describes the background, setup and findings of the experiment. This is achieved by first describing the state of the art and defining further what game feel is. Second, it presents short description of the game that was developed and what methods were used to obtain data from players. Then follows an analysis of the data, and, lastly, this data will be discussed and concluded upon.

\section{State of the Art}
\subsection{Defining Game Feel}
\textit{Game feel} is a relatively unexplored research area. The primary literature on the topic is the book \textit{Game Feel: A Game Designer's Guide to Virtual Sensation} by Swink \cite{swink}. ``Feel" is not meant in a thematic nor emotional/physical sense. Instead it's the kinesthetic sense of manipulating a virtual object --- the sensation of control in a videogame. Talking to game designers, Swink found that game feel is associated with intuitive controls, physical interactions with virtual objects (and the timing and impact of these interactions), as well as aesthetic pleasure and appeal in the form of polishing effects. Having analyzed various games and their components, Swink provided the following definition of game feel: \textit{real-time control of virtual objects in a simulated space, with interactions emphasized by polish} \cite{swink}. These three elements will be discussed further in the following sections.

\subsection{Reacting to the Player's Input}
An important aspect of game feel is controlling virtual objects and how responsive these controls are. Normoyle and J\"{o}rg conducted a study to investigate the relationship between the naturalness of motion (e.g., in the form of realistic and adaptive 3D animations) versus responsiveness (the game reacts instantly to the player's input, regardless of which phase the animation is in) \cite{normoyle_trade-offs_2014}. Developers need to make trade-offs between naturalness and promptness when designing and implementing player controls and animations. This can affect the player's overall sense of control, enjoyment, satisfaction and performance. To test this, Normoyle and J\"{o}rg  created a 3D game with varying degrees of animation blendings \cite{normoyle_trade-offs_2014}. The more natural the animations blend together when moving, the less responsive the character is (e.g., the player has to wait for the virtual character to complete a ``turn-around" animation). While playing, 67 test participants tried the different animations types.  Data was collected by logging the participants' performance in the game, as well as asking them about their experience via a post-questionnaire. Normoyle and J\"{o}rg concluded that one should always prioritize responsiveness, since low responsiveness negatively affects players' perceived ease of use, as well as their objective performances \cite{normoyle_trade-offs_2014}. This knowledge can be tied into the Model Human Processor \cite{card1986model}, implying that games should respond to the player's input within 240 milliseconds in order to appear instant. Also, the computer must provide feedback by displaying images at a rate greater than 10 frames per second in order to maintain the impression of motion. West presented ways to measure response lag and how to minimize it, by understanding the sequence of events that occur from the time the player presses a button, to when the results appear on screen \cite{measure_lag, program_lag}. By using a high-speed camera (60 frames per second) to record both the input device (e.g., a controller) and the screen, it is possible to measure this latency \cite{euro}.

%\subsubsection*{Intuitive and Clear Feedback}
West also provided a good overview of how to implement intuitive non-ambiguous controls in games, by measuring the player's input and comparing it to how the game responds \cite{intuitive_buttons}. An example of this is the ``ghost jump" \cite{ghostJump, canabalt}: players have reached the edge of a platform and decides to jump. However, according to the game's internal physical simulation, the players have already left the ground and is therefore not able to jump. Even though they perceived themselves to be on the ground when pressing the jump button, the game fails to meet this intention. This dissonance between what the players perceived and what actually happened in the simulation negatively affects the game feel --- it's the players' perception that matters. 

%A good-feeling game should afford players whatever they want without having them to think too much about it.

\subsection{Enhancing the Player Experience with Polishing Effects}
Polishing effects, including what some game designers call \textit{juiciness} \cite{juice3}, can be used to make a game's simulation feel more alive. Schell described juciness with the following words: \textit{``When a system shows a lot of second-order motion that a player can easily control, and that gives the player a lot of power and rewards, we say that it is a juicy system --- like a ripe peach, just a little bit of interaction with it gives you a continuous flow of delicious reward."} \cite{schell_art_2008} In other words, the system should give the player continuous feedback for their actions. Game designers Jonasson, Purho and Nijman demonstrated how simple games can feel better by adding layers of effects, such as bouncing motions, screen shake, particles, sounds, impact effects and fluid camera movements, to provide as much visual and auditory flair as possible \cite{juice1, juice2}. Berbece provides similar examples with animation effects \cite{animationSucks}. To some extent, this can be related to the 12 Basic Principles of Animation, which are techniques artists can use to make their animations come to life \cite{animation}. Based on personal experience, game designer Rogers proposed a list of similar techniques that can be utilized in games, such as freezing the animation for a few frames to emphasize a great impact \cite{sticky}.

%On their GitHub page, \cite{juice1} wrote: \textit{``A juicy game feels alive and responds to everything you do tons of cascading action and response for minimal user input.} (\textit{sic})\textit{"} 

\subsection{Designing and Measuring Game Feel}
Game feel consists of real-time control and polishing effects, but there is relatively little academic work about how to design and measure game feel. The current state of designing game feel is that of an art form: it requires game designers to constantly make tiny adjustments in order to make their games feel good \cite{meatboy1, meatboy2, juicyBeast, gameFeelTips}. However, no framework exists to build upon, meaning that game designers practically have to continuously reinvent the wheel. The same goes for the actual game feelings themselves. Even though a player might describe a certain game feeling `floaty', `twitchy' or `responsive', there are no de facto terms that can be used to talk about a specific game feel. It is uncertain when a game goes from being `twitchy' to `floaty'. Game feel consists of many elements, such as graphical presentation, sound, input device, player controls, etc. Further research is needed in order to determine how much each of these elements affect the feel of a game --- and if these findings can be applied universally or only to specific game genres. This paper sets out to investigate one of these aspects, namely the impact of movement controls in 2D platforming games.

\subsection{Experimental Design}
Refer to color paper. Online design. Latin square structure. Anchoring. Etc.

%only sporadic work has been conducted in order to measure game feel \cite{feelBetter}.