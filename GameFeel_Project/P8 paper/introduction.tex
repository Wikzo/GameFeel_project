\section{Introduction}
\citeA{swink} defines that ...

Game design is an artform and craft that requires many years of experience to get right. It's an iterative process wherein designers have to carefully consider and tweak every aspect of a game to get the desired results. One important aspect of game design is \textit{game feel}. Even though game feel is a key element in all videogames, the creation and tweaking of it mostly relies on gut-feeling and extensive testing. There are no official framework or vocabulary that game designers can go to, whether they want to have a game that feels like \textit{Super Mario Bros.} or \textit{Sonic The Hedgehog}. Instead, they will have to "reinvent the wheel", often by means of trial and error.

\cite{swink} defined game feel as \textit{the sensation of control in a videogame}, which is a fundamental building block of game design. Game feel is the extension of the senses, as well as the pleasure of learning and mastering a skill. It's a constant feedback loop between player and system. Game feel happens from the moment a player forms an intention in his head, presses the button on the controller, to when he sees the result of the virtual character jumping on-screen.

Even though game feel is about virtual non-physical experiences, it can still be related to the physical world. Every object in the physical world has properties that define their unique feel, say, their textures, shapes and interactive properties. While a bowling ball feels massive and heavy, a knife feels sharp, pointy and thin. The main problem when talking about game feel is the fact that it's an holistic experience consisting of many contributing elements, such as graphics, physical simulation, sounds, input devices, refresh rate of the screen, etc.

\cite{measure_lag} further argued that \textit{"the 'feel' of a game is in large part described in terms of how 'responsive' it is. Very often a game will be described as 'laggy' or 'sluggish', and by contrast other games will be 'tight' or 'fast'."} However, these are somewhat loose terms, and it is unclear when exactly a game goes from feeling 'sluggish' to 'tight'.

This paper sets out to investigate two elements: A) What types of words players use to describe the feel of a game, and B) What parameters yield those specific descriptions. The project is based on a simple 2D platforming game where players control a rolling ball. Inspired by chapter seven in \cite{swink}, \textit{Response Metrics}, the game changes two types of parameters between each round: how fast the ball accelerates and decelerates when moving horizontally. This pattern is called an ADSR (Attack-Decay-Sustain-Release) envelope, which is often used in order make an electronic musical instrument mimic the sound of a mechanical instrument, e.g., the sound of a pipe organ or a guitar string. This is achieved by modulating the amplitude over time \cite{adsr}.

\section{State of the Art}
\subsection{Defining Game Feel}
\textit{Game feel} is a relatively unexplored research area. The primary literature on the topic is the book \textit{Game Feel: A Game Designer's Guide to Virtual Sensation} by \cite{swink}. "Feel" is not meant in a thematic nor emotional/physical sense. Instead it's the kinesthetic sense of manipulating a virtual object --- the sensation of control in a videogame. Talking to game designers, \citeA{swink} found that game feel is associated with intuitive controls, physical interactions with virtual objects (and the timing and impact of these interactions), as well as aesthetic pleasure and appeal in the form of polishing effects. Having analyzed various games and their components, \cite{swink} provided the following definition of game feel: \textit{real-time control of virtual objects in a simulated space, with interactions emphasized by polish}.

\subsection{Reacting to the Player's Input}
An important aspect of game feel is controlling virtual objects and how responsive these controls are. \cite{normoyle_trade-offs_2014} conducted a study to investigate the relationship between the naturalness of motion (e.g., in the form of realistic and adaptive 3D animations) versus responsiveness (the game reacts instantly to the player's input, regardless of which phase the animation is in). Developers need to make trade-offs between naturalness and promptness when designing and implementing player controls and animations. This can affect the player's overall sense of control, enjoyment, satisfaction and performance. To test this, \cite{normoyle_trade-offs_2014} created a 3D game with varying degrees of animation blendings. The more natural the animations blend together when moving around, the less responsive the character is (e.g., the player has to wait for the virtual character to complete a "turn-around animation"). Showing the different animation types to 67 test participants, they collected data by logging the participants' performance in the game, as well as asking them about their experience via a post-questionnaire. \cite{normoyle_trade-offs_2014} concluded that one should always prioritize responsiveness, since low responsiveness negatively affects players' perceived ease of use, as well as the their objective performances. This knowledge can be tied into the Model Human \cite{card1986model}, implying that games should respond to the player's input within 240 milliseconds in order to appear instant. Also, the computer must provide feedback by displaying images at a rate greater than 10 frames per second in order to maintain the impression of motion. \cite{measure_lag, program_lag} presented ways to measure response lag and how to minimize it, by understanding the sequence of events that occur from the time the player presses a button, to when the results appear on screen.

%\subsubsection*{Intuitive and Clear Feedback}
\cite{intuitive_buttons} also provided a good overview of how to implement intuitive non-ambiguous controls in games, by measuring the player's input and comparing it to how the game  responds. An example of this is the "ghost jump" \cite{ghostJump, canabalt}: a player has reached the edge of a platform and decides to jump. However, according to the game's internal physical simulation, the player has already left the ground and is therefore not able to jump. Even though the player perceived himself to be on the ground when pressing the jump button, the game fails to meet this intention. This dissonance between what the player perceived and what actually happened in the simulation negatively affects the game feel --- it's the player's perception that matters. A good-feeling game should afford players whatever they want without having them to think too much about it.

\subsection{Enhancing the Player Experience with Polishing Effects}
Polishing effects, including what some game designers call \textit{juciness} \cite{juice3}, can be used to make a game's simulation feel more alive. \cite{schell_art_2008} described juciness with the following words: \textit{"When a system shows a lot of second-order motion that a player can easily control, and that gives the player a lot of power and rewards, we say that it is a juicy system --- like a ripe peach, just a little bit of interaction with it gives you a continuous flow of delicious reward."} In other words, the system should give the player continuous feedback for their actions. Game designers \cite{juice1} and \cite{juice2} demonstrated how simple games can feel better by adding layers of effects, such as bouncing motions, screen shake, particles, sounds, impact effects and fluid camera movements, to provide as much visual and auditory flair as possible. On their GitHub page, \cite{juice1} wrote: \textit{"A juicy game feels alive and responds to everything you do tons of cascading action and response for minimal user input.} (\textit{sic})\textit{"} To some extent, this can be related to the 12 Basic Principles of Animation \cite{animation}, which are techniques artists can use to make their animations spring to life. Based on personal experience, game designer \cite{sticky} proposed a list of similar techniques that can be utilized in games, such as freezing the animation for a few frames to emphasize a great impact.

\subsection{Designing and Measuring Game Feel}
Game feel consists of real-time control and polishing effects, but there is relatively little academic work about how to design and measuring game feel. The current state of game feel is that of an art form: it's reserved for game designers that constantly make tiny adjustments in order to make their games feel good \cite{meatboy1, meatboy2, juicyBeast, gameFeelTips}. However, no framework exists to build on, meaning that game designers practically have to "reinvent the wheel" every time. The same goes for the actual game feelings themselves. Even though a player might describe a certain game feeling "floaty", "twitchy" or "responsive", there are no standard go-to terms that can be used to talk about specific game feel; only sporadic work has been conducted in order to measure game feel \cite{feelBetter}. It is uncertain when a game goes from being "twitchy" to "floaty". This is mainly due to game feel being a holistic experience with many contributing factors, such as simulated motions, controls, graphics and sound effects. Further research is needed in order to determine how much each of these elements affects the feel of a game --- and if these findings can be applied universally or only to specific game genres.