\begin{abstract}
The feel of videogames is important, but not very well understood. Game feel is an integral part of game design and can be defined as the moment-to-moment sensation of control in games. It is important for game designers to understand when a game feels a certain way, since it is something that the player is constantly experiencing. Unfortunately, there is a lack of vocabulary for designers to intentionally be able to create a specific game feel. There is a need of a better understanding of why certain games feel like they do, such as what parameters can be used to make a game feeling a particular way. This paper sets out to investigate what words players use to describe the feel of games, as well as what kind of parameters yield these descriptive words. This is achieved using a 2D platforming game in which the response of the player avatar's motion is modulated. The velocity of the avatar changes between rounds, so that the acceleration and deceleration is either slow or fast. This changes the feel of the game. The game was uploaded to the Internet and can be played directly in a web browser. The game was shared on various social media and gaming communities. 274 participants played the game. Between each round, players were asked to describe their perceived feel of controlling the avatar, as well as rate it in categories such as how `fluid', `floaty' and `responsive' the game was. It was found that players have a hard time distinguishing between the velocities when the response time is below 240 milliseconds. However, when the acceleration and deceleration took more than 240 milliseconds, players reported the controls to feel `heavy' and `stiff'. Players reacted more strongly to slow acceleration than deceleration. Whether these findings apply to other game genres is yet to be investigated.
\end{abstract}
