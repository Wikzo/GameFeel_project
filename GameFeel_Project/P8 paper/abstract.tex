\begin{abstract}
The feel of videogames is important, but not very well understood. Game feel is an integral part of game design and is the moment-to-moment sensation of control in games. Since it is something the player is constantly experiencing, it is important to be able to define when a game feels `twitchy' or `floaty'. There is a lack of vocabulary for designers and players to be able to talk about game feel. This paper sets out to investigate what words players use to describe the feel of a game, as well as what kind of parameters yield these words. This is achieved using a 2D platforming game in which the response of the player avatar motion is modulated. The velocity of the avatar changes between rounds, so that the acceleration and deceleration is either quick or slow. This changes the feel of the game. Players were asked to describe these feelings, as well as rate them in categories such as how `fluid', `floaty' and `responsive' the game felt. It was found that players have a hard time distinguishing between the velocities when the response time is below 250 ms. However, when the acceleration and deceleration took more than 250 ms, players reported the controls to feel very unresponsive and stiff. It was found that the players react more strongly to slow acceleration than deceleration. Whether these findings apply to other game genres is yet to be investigated.

\end{abstract}