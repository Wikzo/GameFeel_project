\begin{abstract}
The feel of videogames is important, but not very well understood. Game feel is an integral part of game design and can be defined as the moment-to-moment sensation of control in games. It is important for game designers to understand when a game feels a certain way, since it is something that the player is constantly experiencing. There is a need of a better understanding of why certain games feel like they do, such as which parameters can be used to make a game feeling a particular way. This paper sets out to investigate what words players use to describe the feel of games, as well as what kind of parameters yield these descriptive words. This is attempted by using a 2D platforming game in which the response of the player avatar's motion is modulated. Between each round, players were asked to describe their perceived feel of controlling the avatar. The majority used basic words to describe the feel of the game, such as `heavy', `slow', `responsive' and `realistic'. While some participants were quite sensitive to small changes, others expressed that they couldn't feel any differences. Even though all participants agreed that the feel of games is essential, there still seems to be a lack of understanding behind what game feel is. Further research is needed to investigate the influence of other factors, such as game genre, graphics, sounds, level design and player attention.
\end{abstract}
