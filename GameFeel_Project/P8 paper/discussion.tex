\section{Discussion \& Conclusion} \label{discussion}
When it comes to how people describe game feel in their own words, it seems that many use basic descriptions such as `heavy', `slow', `responsive', `realistic', `sluggish' and `floaty'. For the type of game used in this project, those description might be enough to describe the game feel.

%There are a few correlations between the acceleration, deceleration and the pre-defined words. However, it seems like there are different interpretations of what those words mean, e.g., what it means for a game to feel \textit{stiff}.

While some participants seemed quite sensitive to even smaller changes, others reported that they didn't feel any difference between the rounds.

The curves presented in this paper are based on averages. The results are thus less clear when compared to those provided by Swink (see Section \ref{design}). In this experiment, participants were asked to rate the acceleration and deceleration values along a group of Likert scales. An alternative approach would be to use A/B testing and ask participants whether stimulus A felt more or less `twitchy' than stimulus B. It would also be interesting to let players themselves tweak and tune the game's parameters in order to achieve what they think feels best.

Furthermore, one could look into more evenly-distributed sequences of the stimuli, say, within the 240 ms interval. Also, it might be interesting to look into non-linear curves and the influence of how responsive this feels (e.g., a slow acceleration that starts with an initial bump in its curve).

It appears that some participants were confused by the term \textit{game feel}. This is to be expected, since it's still a loose and relatively unknown concept about how players perceive games. A similar term, \textit{mouthfeel}, describes, as the name suggests, the sensation of food in the mouth. Even though the term was coined in 1951 \cite{mouthfeel}, it is still a relatively unknown concept. It might take a while before game feel finds its place in the vocabulary of the common game player.

Putting words on the feel of anything is difficult, as one participant expressed in the post-questionnaire: \textit{``Sorry if my answers weren't super specific. I was having a bit of a hard time finding the words."} Another participant wrote that `feel' is a very non-specific word: \textit{``I think this is the survey of someone who is very much immersed in `game speak'. Outside of games, `feel' is a very non-specific word, and is used for communicating all the stuff that is hard to communicated. I come from a design background and to me `feel' is used when describing the overall feeling I get from a thing when I can't pinpoint why I get that feeling (like, a strawberry is just cute, and I can't say directly why)."} Additionally, a different participant pinpointed that the feel depends on the context: \textit{``I think the feel depends on the game. If we control a soccer ball on a platformer, faster and more responsive controls make sense, but if it was a bowling ball, it would be better to be a little more slow and `draggy', as long as the levels and obstacles are designed with this in mind."}

Game feel is a holistic experience with many contributing factors. In this experiment, only the avatar movement was considered. However, all other elements still indirectly influence the overall game feel, such as the rolling animation of the ball; the gravity and jumping mechanic; the level design; and the input device. It is unclear how big an influence the altering acceleration/deceleration has over these other factors, as well as other contributing elements such as the attention, mood and motivation of the player. Further research into the influence of these is required.

\section{Accompanying Video}
A video has been put together in order to describe game feel and this project. The video is available on YouTube: \url{http://youtu.be/S-EmAitPYg8}.