\section{Discussion \& Future Work} \label{discussion}
It appears that some participants were confused by the term \textit{game feel}. This is to be expected, since it's still a loose and relatively unknown concept about how players perceive games. When players had to describe the game feel with their own words, it seems that many use basic descriptions such as `heavy', `slow', `responsive', `realistic', `sluggish' and `floaty'. For the type of game used in this project, those descriptions might be enough to describe the game feel. While some participants seemed quite sensitive to even smaller changes, others reported that they didn't feel any difference between the rounds.

In this experiment, participants were asked to rate the acceleration and deceleration times along a group of Likert scales. An alternative approach would be to use A/B testing and ask participants whether stimulus A felt more or less `twitchy' than stimulus B. It would also be interesting to let players tweak the game's parameters themselves, in order to achieve what they think feels best.

Furthermore, one could look into evenly-distributed sequences of the stimuli, e.g., within the 240 ms interval. It might also be interesting to look into non-linear curves and the influence of how responsive these curves might feel (e.g., a slow acceleration that starts with an initial bump in its curve).

Putting words on the feel of anything is difficult. One participant wrote that `feel' is a very non-specific word: \textit{``Outside of games, `feel' is a very non-specific word, and is used for communicating all the stuff that is hard to communicate. I come from a design background and to me `feel' is used when describing the overall feeling I get from a thing when I can't pinpoint why I get that feeling (like, a strawberry is just cute, and I can't say directly why)."}

Game feel is a holistic experience with many contributing factors. In this experiment, only the avatar movement was considered. However, all other elements indirectly influence the overall game feel, such as the rolling animation of the ball; the gravity and jumping mechanic; the level design; and the input device. It is unclear how big an influence the altering acceleration/deceleration had over these other factors, as well as other contributing elements such as the attention, mood and motivation of the player, as well as cultural differences (e.g., a player from the Western world might perceive game feel in a different way than a player from Asia. Further research into the influence of these is required.
%\section{Accompanying Video}
%A video has been put together in order to describe game feel and this project: \url{http://youtu.be/S-EmAitPYg8}.