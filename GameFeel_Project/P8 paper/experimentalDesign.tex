\section{Experimental Design} \label{experimentalDesign}
%\subsection{Research Question}
%How does the acceleration and deceleration of a player avatar influence the game feel of a 2D platformer? What types of words do players use to describe the game feeling, and what parameters result in these words?
%\subsection{Stimuli}
%The time it took for the player avatar to accelerate and decelerate changed between rounds.

\subsection{Stimulus Presentation}
Participants played four rounds of the game. Each round had different acceleration and deceleration values. All other factors were held constant, e.g., the level design, sound effects and the parameters for jumping. Between rounds, participants were asked to describe how it felt to play the game. The term \textit{game feel} was explicitly not explained, so that participants would try to describe the feel of the game from their own understanding of what game feel is.

The experiment was designed as a repeated-measures, within-participant design \cite{cunningham}. This means that the participants were exposed to the stimuli (the changing acceleration and deceleration) multiple times. Additionally, each participant would see all of the available stimuli (each combination within the two categories, \textit{fast} and \textit{slow}), thereby acting as their own control group by comparing the different stimuli to each other.

\subsubsection{Latin Squares} \label{latinSection}
There are four possible combinations, as shown in Table \ref{tab:combinations}. One of the strengths with within-participant designs is that it doesn't require as many participants as a between-participant design, since each participant will try all the conditions. However, one disadvantage of within-participant designs is the risk of carry-over effects \cite{experimental1}. This might be due to fatigue (e.g., the participants become bored after having experienced the multiple conditions) or practice (e.g., the participants are better at the end than when they started).

\begin{table} \centering
\caption{Four different combinations.}
\label{tab:combinations}
\begin{tabular}{ccc}
\toprule
& \textbf{Acceleration} & \textbf{Deceleration} \\
\midrule
\textbf{Stimulus 1} & Fast (A) & Fast (A)\\
\textbf{Stimulus 2} & Slow (B) & Slow (B)\\
\textbf{Stimulus 3} & Fast (A) & Slow (B)\\
\textbf{Stimulus 4} & Slow (B) & Fast (A)\\
\bottomrule
\end{tabular}
\end{table}

The order in which the different stimuli are shown can affect the participant's behaviour/perception. A way to prevent this is to use a counter-balanced design. This method reduces the risks of the order influencing the results \cite{experimental2}. Ideally, since there are four possible conditions (see Table \ref{tab:combinations}), there should be 4x3x2x1 different orders, i.e., 24 orders of treatment. The number of participants must also be a multiple of 24, since there should be an equal number in each group \cite{experimental2}. Having 24 different combinations was deemed too complex; thus, an incomplete balanced design in the form of Latin squares was used instead (see Table \ref{table:latin}). Even though the order effects aren't eliminated completely, they become balanced.

\begin{table} \centering
\scriptsize
\caption{Latin squares are arranged in rows and columns such that each of the stimuli conditions only occur once in each row and column. The first letter is the acceleration, the second letter is the deceleration. `A' means fast and `B' means slow.}
\label{table:latin}
\begin{tabular}{ccccc}
\toprule
& \textbf{Stimulus 1} & \textbf{Stimulus 2} & \textbf{Stimulus 3} & \textbf{Stimulus 4}\\
\midrule
\textbf{Seq. 1} & AA & BB & BA & AB\\
\textbf{Seq. 2} & BB & AB & AA & BA\\
\textbf{Seq. 3} & AB & BA & BB & AA\\
\textbf{Seq. 4} & BA & AA & AB & BB\\
\bottomrule
\end{tabular}
\end{table}

\subsection{Task} \label{task}
Taking inspiration from Section \ref{marioLevel}, a questionnaire was built into the game. Initially, participants were asked to fill in basic demographical information. Before the game started, participants didn't know what to expect: they didn't know the range of what stimuli they would see, hence, they might be hesitant to use the extreme values on the Likert scales in the questionnaire. To counter this, participants were presented with two examples of the conditions (very \textit{fast} and very \textit{slow} acceleration/deceleration) in a closed environment (see Figure \ref{fig:example}). This is called \textit{anchoring} \cite{cunningham} and gives participants a common reference point of of what to expect in the game. However, the game doesn't explicitly describe the two examples, i.e., stating that it's the acceleration/deceleration that change.

\begin{figure}[htbp]
\centering
\includegraphics[width=0.4\textwidth]{Pics/example}
\caption{Participants were shown two examples of the extreme conditions before playing the actual game.}
\label{fig:example}
\end{figure}

Afterwards, the game begins and players are asked to find and collect three stars. The purpose of the stars is to ensure that players move/jump around enough in order to experience the feel of controlling the ball. The stars have been placed in the beginning, middle and end of the level, so players have to experience the whole level each time. The level consists of traditional platforming elements, as well as obstacles in the form of a few moving enemies and spikes. Figure \ref{fig:level} shows an overview of the game's level.

\begin{figure*}[htbp]
\centering
\includegraphics[width=1\textwidth]{Pics/levelStructure}
\caption{Participants played the same level four times.}
\label{fig:level}
\end{figure*}

Each time players collect three stars, the game is paused and a questionnaire is shown (see Figure \ref{fig:questionnaire}). The questionnaire consists of three sets of questions. The first asks players to try and describe the feeling of controlling the ball on the ground and in the air, with their own words. Inspired by Section \ref{colour}, it is stressed that the chosen word(s) should be something that the player would use to describe the feeling to a friend. As in Section \ref{colour}, there were no restrictions on what types of words players could use (the input field's length is equivalent to approximately 66 characters, but it is possible to scroll forward/backward if a participant writes more than this). Each input field includes five randomly-chosen example words to give participants an idea of what could be used (see Table \ref{table:wordsExamples}). Participants were asked to describe the feeling both on the ground and in air. Even though only horizontal movement is changed in the form of the acceleration and deceleration (both apply on ground and in air), there is a possibility that players perceived the game feel differently on ground and in air. Instead of trying to describe both at the same time, players were shown two input fields.

\begin{table} \centering
\scriptsize
\caption{Participants were shown five randomly-chosen words to give them an idea of what they could write.}
\label{table:wordsExamples}
\begin{tabular}{ccccc}
\toprule
Fragile & Rigid & Firm & Solid & Thick\\
Fixed & Robust & Sore & Steadfast & Wild\\
Constant & Free & Hard & Tough & Restricted\\
Limited & Reduced & Fast & Heavy & Slow\\
Enjoyable & Stressful & Annoying & Realistic & Normal\\
Difficult & Easy & Dry & Juicy & Mechanical\\
Automatic & Organic & Exciting & Wet & Simple\\
Complicated & Direct & Inert & Unrealistic & Light\\
\bottomrule
\end{tabular}
\end{table}

After describing the game feel with the players' own words, a new set of questions was shown. Here, players were asked to rate the game feel on a 7-point Likert scale. Taking inspiration from Swink's book \cite{swink}, players rated the game feeling on how \textit{twitchy}, \textit{fluid}, \textit{stiff}, \textit{floaty} and \textit{responsive} they felt the controls were. Lastly, players were asked about how \textit{enjoyable}, \textit{difficult} and \textit{frustrating} it was to control the ball, as well as \textit{how much they liked} the control of the ball. In case players forgot how it felt to control the ball, they could always click on the \textit{Resume Playing} button to refresh their memories before continuing with the questionnaire.

After finishing the fourth round, participants were asked to complete an online post-questionnaire. The questions here were about game feel in general: \textit{What parameters do you think changed between each round in the ball game?} and \textit{In your own words, how would you define the feel of games?}. Additionally, participants were asked to describe the feel of six other platforming games: \textit{Mega Man}, \textit{LittleBigPlanet}, \textit{Donkey Kong}, \textit{Super Meat Boy}, \textit{Prince of Persia} (1989) and \textit{Super Mario Bros.}

\begin{figure*}[htbp]
\centering
\includegraphics[width=0.95\textwidth]{Pics/game_phases}
\caption{When the player collected three stars, the game paused and showed a questionnaire.}
\label{fig:questionnaire}
\end{figure*}

\subsection{Participants}
The game was uploaded to a server and shared on social media websites and gaming forums, such as NeoGAF, N-club, Play:Right, the Unity Community, , Rock Paper Shotgun, Nintendo Life, 3D Buzz, Gamasutra and Spiludvikling.dk. The primary target were people who already play videogames; however, others were welcome to play the game as well. The participants were oblivious to the exact purpose and methods used in the experiment; they only knew that the research topic was about game feel, but not how this was measured.

A landing page\footnote{The game is available here: \\ \url{http://tunnelvisiongames.com/g/GameFeel.html}} was created where participants could choose to either play the game in their browser or download a standalone build. This page also ensured that all of the participants would read the exact same description of the experiment before beginning. A \$10 gift card for either Amazon or Steam was promised to one randomly-chosen participant. Also, Bitly \cite{bitly} was used to keep track of incoming traffic to the website.

Whenever a player completed a round (collecting three stars and answering the questionnaire), data was  sent to an MySQL database. The data entries include demographical information (age, gender, region, previous experience with games), parameter information (acceleration and deceleration times) and player descriptions (how players described the game feel, and how they rated the game feel on the pre-defined words). Target platform (web player or Windows player), player death count, average framerate and time spent on the level were also saved.

To ensure the order in the Latin square (see Section \ref{latinSection}), players were assigned a number between 1 and 4 when starting the game. This number corresponds to the sequences in Table \ref{table:latin}. This was achieved by taking modulus 4 of the total amount of participants having completed the experiment and adding 1 to it. For instance, if 26 participants had played the game before entering, the next player would be assigned the sequence number (26 \% 4) + 1 = 3.