\documentclass{acm_proc_article-sp}	
%\bibliographystyle{apalike} %% <---
\pagenumbering{arabic}
\usepackage{url}
\usepackage{enumitem}
\usepackage{url}
%\usepackage{hyperref}
% nicer tables: booktab
\usepackage{booktabs}
\setlength{\abovetopsep}{1ex}
\setlength{\textfloatsep}{0.3cm}

% remove copyright stuff
%\makeatletter
%\def\@copyrightspace{\relax}
%\makeatother


\begin{document}
%
% --- Author Metadata here ---
\conferenceinfo{MindTrek}{'15 Finland, Stuff}
\CopyrightYear{2015} % Allows default copyright year (20XX) to be over-ridden - IF NEED BE.
\crdata{0-12345-67-8/90/01}  % Allows default copyright data (0-89791-88-6/97/05) to be over-ridden - IF NEED BE.
% --- End of Author Metadata ---


%\title{Designing and Measuring Game Feel in\\2D Platforming Games}


  \title{%
  %Measuring Game Feel And The Impact of Control\\
  %Measuring the Impact of Player Control on Game Feel\\
  %Measuring the Impact of Modulating Control on Game Feel
  Measuring How Game Feel Is Influenced by\\the Player Avatar's Acceleration and Deceleration\\
  \large Using a 2D Platformer to Describe Players' Perception of Controls in Videogames}





\numberofauthors{2}
\author{
\alignauthor
Gustav Dahl\\
       \affaddr{Aalborg University, Denmark}\\
       \email{gdahl11@student.aau.dk}
\alignauthor
Martin Kraus\\
       \affaddr{Aalborg University, Denmark}\\
       \email{martin@create.aau.dk}
}


\maketitle

\begin{abstract}
Version 6. Abstract goes here. Abstract goes here. Abstract goes here.

\textbf{Motivation}: Context, Need, task, Objective of the document
\textbf{Outcome}: Findings, Conclusion, Perspectives
\end{abstract}

% A category with the (minimum) three required fields
%\category{H.4}{Information Systems Applications}{Miscellaneous}
%A category including the fourth, optional field follows...
%\category{D.2.8}{Software Engineering}{Metrics}[complexity measures, performance measures]

%\terms{Experimentation}

\keywords{Game design, game feel, game development, perception} % NOT required for Proceedings

\section{Introduction}
Game design is a difficult artform that requires many years of experience to get right. It's an iterative process wherein designers have to carefully consider and tweak every aspect of a game to get the desired results, including the feel of the game. Even though \textit{game feel} is a key element in all videogames, the creation and tweaking of it mostly relies on the designer's gut feeling, as well as extensive testing. There are no framework or vocabulary that game designers can build upon, whether they want to have a game that feels like \textit{Super Mario Bros.} or \textit{Sonic The Hedgehog}. Instead, designers will have to reinvent the wheel, often by means of trial and error.

One definition of game feel is proposed by \cite{swink} as \textit{the sensation of control in a videogame}, which is a fundamental building block of game design. Game feel is the extension of the senses, as well as the pleasure of learning and mastering a skill. It's a constant feedback loop between player and system. Game feel happens from the moment a player forms an intention in his head and presses a button, to when he sees the result of the virtual character jumping on-screen.

Even though game feel is about virtual non-physical experiences, it can still be related to the physical world. Every object in the physical world has properties that define their unique feel, say, their textures, shapes and interactive properties. While a bowling ball feels massive and heavy, a knife feels sharp, pointy and thin. When it comes to games, designers and players don't have the required vocabulary to talk about these kinds of things. One of the main problems when discussing game feel is the fact that it's an holistic experience consisting of many contributing elements, such as graphics, physical simulation, sounds, input devices, refresh rate of the screen, etc. It is difficult to isolate the impact of each of these aspects.

\cite{measure_lag} further argued that \textit{``the 'feel' of a game is, in large part, described in terms of how responsive it is. Very often a game will be described as 'laggy' or 'sluggish', and by contrast other games will be 'tight' or 'fast'."} However, these are somewhat loose terms, and it is unclear when exactly a game goes from feeling 'sluggish' to 'tight'.

This paper sets out to investigate two things: A) What types of words players use to describe the feel of a game, and B) What parameters yield those specific descriptions. The project is based on a simple 2D platforming game where players control a rolling ball with the keyboard. Inspired by chapter seven in \cite{swink} (\textit{Response Metrics}), the game changes two types of parameters between each round: how fast the ball accelerates and how fast it decelerates when moving horizontally. This pattern is inspired by ADSR envelopes (Attack-Decay-Sustain-Release) \cite{adsr}, which are often used in order make an electronic musical instrument mimic the sound of a mechanical instrument, e.g., the sound of a pipe organ or a guitar string. This is achieved by modulating the amplitude over time.

The game was put on the Internet where people could play it in their browser. Players had to play the same game four times, where each round changed the acceleration and deceleration. Between the rounds, players would try to describe the feel of the controls, as well as rate how strongly they felt this/these feeling(s). X number of people participated.

This paper describes the findings from the test. First, by describing the state of the art and defining further what game feel is. Second, a short description of the game itself and what methods were used to obtain data. Then follows analysis of the data. Lastly, a conclusion and discussion.

\section{State of the Art}
\subsection{Defining Game Feel}
\textit{Game feel} is a relatively unexplored research area. The primary literature on the topic is the book \textit{Game Feel: A Game Designer's Guide to Virtual Sensation} \cite{swink}. ``Feel" is not meant in a thematic nor emotional/physical sense. Instead it's the kinesthetic sense of manipulating a virtual object --- the sensation of control in a videogame. Talking to game designers, \citeA{swink} found that game feel is associated with intuitive controls, physical interactions with virtual objects (and the timing and impact of these interactions), as well as aesthetic pleasure and appeal in the form of polishing effects. Having analyzed various games and their components, \cite{swink} provided the following definition of game feel: \textit{real-time control of virtual objects in a simulated space, with interactions emphasized by polish}.

\subsection{Reacting to the Player's Input}
An important aspect of game feel is controlling virtual objects and how responsive these controls are. \cite{normoyle_trade-offs_2014} conducted a study to investigate the relationship between the naturalness of motion (e.g., in the form of realistic and adaptive 3D animations) versus responsiveness (the game reacts instantly to the player's input, regardless of which phase the animation is in). Developers need to make trade-offs between naturalness and promptness when designing and implementing player controls and animations. This can affect the player's overall sense of control, enjoyment, satisfaction and performance. To test this, \cite{normoyle_trade-offs_2014} created a 3D game with varying degrees of animation blendings. The more natural the animations blend together when moving, the less responsive the character is (e.g., the player has to wait for the virtual character to complete a ``turn-around" animation). Showing the different animation types to 67 test participants, they collected data by logging the participants' performance in the game, as well as asking them about their experience via a post-questionnaire. \cite{normoyle_trade-offs_2014} concluded that one should always prioritize responsiveness, since low responsiveness negatively affects players' perceived ease of use, as well as the their objective performances. This knowledge can be tied into the Model Human Processor \cite{card1986model}, implying that games should respond to the player's input within 240 milliseconds in order to appear instant. Also, the computer must provide feedback by displaying images at a rate greater than 10 frames per second in order to maintain the impression of motion. \cite{measure_lag, program_lag} presented ways to measure response lag and how to minimize it, by understanding the sequence of events that occur from the time the player presses a button, to when the results appear on screen.

%\subsubsection*{Intuitive and Clear Feedback}
\cite{intuitive_buttons} provided a good overview of how to implement intuitive non-ambiguous controls in games, by measuring the player's input and comparing it to how the game  responds. An example of this is the ``ghost jump" \cite{ghostJump, canabalt}: a player has reached the edge of a platform and decides to jump. However, according to the game's internal physical simulation, the player has already left the ground and is therefore not able to jump. Even though the player perceived himself to be on the ground when pressing the jump button, the game fails to meet this intention. This dissonance between what the player perceived and what actually happened in the simulation negatively affects the game feel --- it's the player's perception that matters. A good-feeling game should afford players whatever they want without having them to think too much about it.

\subsection{Enhancing the Player Experience with Polishing Effects}
Polishing effects, including what some game designers call \textit{juciness} \cite{juice3}, can be used to make a game's simulation feel more alive. \cite{schell_art_2008} described juciness with the following words: \textit{``When a system shows a lot of second-order motion that a player can easily control, and that gives the player a lot of power and rewards, we say that it is a juicy system --- like a ripe peach, just a little bit of interaction with it gives you a continuous flow of delicious reward."} In other words, the system should give the player continuous feedback for their actions. Game designers \cite{juice1} and \cite{juice2} demonstrated how simple games can feel better by adding layers of effects, such as bouncing motions, screen shake, particles, sounds, impact effects and fluid camera movements, to provide as much visual and auditory flair as possible. To some extent, this can be related to the 12 Basic Principles of Animation, which are techniques artists can use to make their animations spring to life \cite{animation}. Based on personal experience, game designer \cite{sticky} proposed a list of similar techniques that can be utilized in games, such as freezing the animation for a few frames to emphasize a great impact.

%On their GitHub page, \cite{juice1} wrote: \textit{``A juicy game feels alive and responds to everything you do tons of cascading action and response for minimal user input.} (\textit{sic})\textit{"} 

\subsection{Designing and Measuring Game Feel}
Game feel consists of real-time control and polishing effects, but there is relatively little academic work about how to design and measure game feel. The current state of game feel is that of an art form: it's reserved for game designers that constantly make tiny adjustments in order to make their games feel good \cite{meatboy1, meatboy2, juicyBeast, gameFeelTips}. However, no framework exists to build upon, meaning that game designers practically have to reinvent the wheel every time. The same goes for the actual game feelings themselves. Even though a player might describe a certain game feeling 'floaty', 'twitchy' or 'responsive', there are no standard go-to terms that can be used to talk about specific game feel; only sporadic work has been conducted in order to measure game feel \cite{feelBetter}. It is uncertain when a game goes from being 'twitchy' to 'floaty'. This is mainly due to game feel being a holistic experience with many contributing factors, such as simulated motions, controls, graphics and sound effects. Further research is needed in order to determine how much each of these elements affects the feel of a game --- and if these findings can be applied universally or only to specific game genres.
\section{Design \& Implementation} \label{design}
\subsection{Modulating Acceleration and Deceleration}
Swink discussed different ways to modulate the player avatar movement in the chapter \textit{Response Metrics} \cite{swink}. Inspired by ADSR envelopes (Attack-Decay-Sustain-Release), which are often used in order make an electronic musical instrument mimic the sound of a mechanical instrument \cite{adsr}, he proposed the idea of using velocity modulation to change the game feel. By altering the attack and release phase (or, acceleration and deceleration), it is possible to create different game feel, as illustrated by Figures \ref{fig:adsr_stiff} and \ref{fig:adsr_loose}.

%, e.g., the sound of a pipe organ or a guitar string. This is achieved by modulating the amplitude over time.

%Inspired by Swink's discussion about \textit{Response Metrics} \cite{swink}, a 2D platforming game was developed. The game changes two types of parameters between each round: how fast the ball accelerates and how fast it decelerates when moving horizontally. Swink calls these the \textit{attack} and \textit{release} phases, or, the acceleration and deceleration of avatar movement. Hence, the velocity of the player's avatar is modulated over time, creating different types of game feel. Figures \ref{fig:adsr_stiff} and \ref{fig:adsr_loose} show two examples of the modulations proposed by Swink.


%According to Swink, when the acceleration or deceleration is very long (e.g., the avatar takes more than 100 milliseconds to be perceived to be moving), the impression of instantaneous response erodes. Even if there are small changes in the velocity, if these cannot be perceived by the player, the game might feel unresponsive \cite{swink}. This is illustrated by Figure \ref{fig:adsr}.

\begin{figure}[htbp]
\centering
\includegraphics[width=0.30\textwidth]{Pics/adsr_stiff}
\caption{Short acceleration/deceleration gives a responsive, but stiff, feel. Figure inspired by Swink \cite{swink}.}
\label{fig:adsr_stiff}
\end{figure}

\begin{figure}[htbp]
\centering
\includegraphics[width=0.30\textwidth]{Pics/adsr_loose}
\caption{Long acceleration gives a loose, but fluid, feel. Figure inspired by Swink \cite{swink}.}
\label{fig:adsr_loose}
\end{figure}

Taking inspiration from Swink, a game was developed with this concept in mind. The game changes two types of parameters between each round: how fast the ball accelerates and how fast it decelerates (when moving horizontally). Hence, the velocity of the player's avatar is modulated over time. This means that when the player presses the movement button, the ball will take a certain amount of time before it reaches its maximum velocity. The same is applied when the player releases the button: the ball will gradually slow down, until it stops.


\begin{figure}[htbp]
\centering
\includegraphics[width=0.30\textwidth]{Pics/response}
\caption{Model of response time and player perception. Figure inspired by Swink \cite{swink}.}
\label{fig:response}
\end{figure}
%Depending on how big a delay there is from the player triggering an event to getting feedback, the game will gradually feel less responsive.

Two intervals were chosen, inspired by Swink's model of player perception and feedback (see Figure \ref{fig:response}). The first interval, \textit{fast}, is between 1 millisecond and 240 milliseconds (staying within the limits of real-time perception). The second interval, \textit{slow}, is from 241 milliseconds to 1500 milliseconds (see Table \ref{tab:time}). For each round in the game, the player is assigned randomly-chosen time values within the two intervals. The reason for choosing random values instead of fixed values is that it isn't perfectly clear at what exact point a game goes from feeling responsive to unresponsive (it's a gradual scale, as shown in Figure \ref{fig:response}). Additionally, the model in Figure \ref{fig:response} depicts the perception of getting discrete feedback, e.g., clicking on a button turns on a light bulb after 50 milliseconds. In the case of avatar movement, there is a continuous stream feedback while the player is holding the movement button, since the avatar is gradually moving forward. However, if the acceleration/deceleration time is very big, the avatar will take some time before it gathers a velocity that can be perceived by the player. In other words, the time values change the total amount of time it takes from when a player presses a button to when the avatar reaches its maximum velocity (or, when releasing the button, reaches a velocity of zero). The acceleration/deceleration is thus scaled depending on the time values, using Equation \ref{eq:erl}.
\begin{equation} \label{eq:erl} %% source: http://www.calculatorsoup.com/calculators/physics/velocity_a_t.php
a = (v - v_0)/t
\end{equation} 
where $a$ is the acceleration/deceleration, $v$ is the target (maximum) velocity, $v_0$ is the initial velocity and $t$ is the time.

%%% This is how to insert a table %%%
\begin{table}[htbp]
\small
\centering
\begin{tabular}{|c|c|}
\hline \textbf{Fast}
& \textbf{Slow}\\\hline
[1 ms - 240 ms]
& [241 ms - 1500 ms]
\\\hline
\end{tabular}
\caption{Time intervals for the acceleration/deceleration.}
\label{tab:time}
\end{table}

The game features other parameters, such as friction, gravity, jump velocity and the aforementioned ghost jump, but only the horizontal ground acceleration and deceleration changed between rounds. The values for these other parameters can be found in APPENDIX A.

The game was developed using the Unity game engine, which allows for exporting to both web and standalone platforms. The game was released for web\footnote{Unity recently made it possible to export to the WebGL platform, but at the time of writing there are bugs and performance issues, so it was chosen to use the standard web player using a browser plug-in.} and Windows. It's a traditional 2D side-scrolling platform game in which players move a small soccer ball from left to right to collect three stars. The game is controlled with the arrow keys and the spacebar.

\subsection{Experimental Design}
The experiment has been designed as a repeated-measures, within-participant design \cite{cunningham}. This means that the participants will be exposed to the stimuli (the changing acceleration and deceleration) multiple times. Additionally, each participant will see all of the available stimuli, thereby acting as their own control group by comparing the different stimuli to each other.

\subsubsection{Latin Squares} \label{latinSection}
There are four possible combinations, as shown in Table \ref{tab:combinations}. One of the strengths with within-participant designs is that it doesn't require as many participants as a between-participant design, since each participant will try all the conditions. However, one disadvantage of within-participant designs is the risk of carry-over effects \cite{experimental1}. This might be due to fatigue (e.g., the participants become bored after having experienced the multiple conditions) or practice (e.g., the participants are better at the end than when they started).

%%% This is how to insert a table %%%
\begin{table}[htbp]
\small
\centering
\begin{tabular}{|c|c|c|}
\hline  & \textbf{Acceleration}
& \textbf{Deceleration}\\\hline
\textbf{Stimulus 1} & Fast (A)
& Fast (A)
\\\hline
\textbf{Stimulus 2} & Slow (B)
& Slow (B)
\\\hline
\textbf{Stimulus 3} & Fast (A)
& Slow (B)
\\\hline
\textbf{Stimulus 4} & Slow (B)
& Fast (A)
\\\hline
\end{tabular}
\caption{Four different combinations.}
\label{tab:combinations}
\end{table}

The order in which the different stimuli is shown can affect the participant's behaviour.  A way to prevent this is to use a counter-balanced design. This method reduces the chances of the order influencing the results \cite{experimental2}. Ideally, since there are four possible conditions (see Table \ref{tab:combinations}), there should be 4x3x2x1 different orders, i.e., 24 orders of treatment. The number of participants must also be a multiple of 24, since there should be an equal number in each group \cite{experimental2}. This was deemed to complex, so instead an incomplete balanced design in the form of Latin squares have been used (see Table \ref{table:latin}). Even though the order effects aren't eliminated completely, they become balanced. Latin squares are arranged in rows and columns such that each of the stimuli conditions only occur once in each row and column. For experiments with an even number of conditions, the first row of a Latin square follows the formula $1, 2, n, 3, n-1, 4, n-2...$, where $n$ is the number of conditions. For subsequent rows, 1 is added to the previous and returning to 1 after $n$ \cite{experimental2}. Using this method, only four different orders are needed. This means that the first player will experience order 1, the second player order 2, etc.

%%% This is how to insert a table %%%
\begin{table}[htbp]
\scriptsize
\centering
\begin{tabular}{|c|c|c|c|c|}
\hline
 & \textbf{Stimulus 1} & \textbf{Stimulus 2} & \textbf{Stimulus 3} & \textbf{Stimulus 4}\\
 \hline
\textbf{Seq. 1} & AA & BB & BA & AB\\
\hline
\textbf{Seq. 2} & BB & AB & AA & BA\\
\hline
\textbf{Seq. 3} & AB & BA & BB & AA\\
\hline
\textbf{Seq. 4} & BA & AA & AB & BB\\
\hline
\end{tabular}
\caption{Latin square showing the four possible sequences. The first letter is the acceleration, the second letter is the deceleration. A means fast and B means slow.}
\label{table:latin}
\end{table}

%\subsection{The Game Itself}

\subsubsection{Structure of the Game}
Taking inspiration from Section \ref{marioLevel}, a questionnaire was built into the game. Before the game begins, players are asked to fill in basic demographical information. After this, players are presented with two examples of the extreme conditions (very fast and very slow acceleration/deceleration). This is called \textit{anchoring} \cite{cunningham} and gives players an idea of what to expect in the game, without explicitly telling them what the examples consists of. Afterwards, the game starts and and players are asked to find and collect three stars. The purpose of the stars is to ensure that players move/jump around enough in order to experience the feel of controlling the ball. The stars have been placed in the beginning, middle and end of the level, so players have to experience the whole level each time. The level consists of traditional platforming elements, as well as obstacles in the form of a few moving enemies and spikes. Figure \ref{fig:level} shows an overview of the game's level.

\begin{figure*}[htbp]
\centering
\includegraphics[width=1\textwidth]{Pics/levelStructure}
\caption{Participants play the same level four times.}
\label{fig:level}
\end{figure*}

Each time players collect three stars, the game is paused and a questionnaire is shown (see Figure \ref{fig:questionnaire}). The questionnaire consists of three sets of questions. The first asks players to try and describe the feeling of controlling the ball on the ground and in the air. Inspired by Section {colour}, it is stressed that the chosen word(s) should be something that the player would use to describe the feeling to e.g. a friend. As in Section \ref{colour}, there are no restriction on what types of words players use. In the input box, a few examples are shown to give players an idea of what could be used. These words are chosen randomly and includes adjectives such as `rigid', `reduced', `slow', `dry', `mechanical' and `juicy'. Players are asked to describe the feeling both on the ground and in air. Even though only horizontal movement is changed in the form of the acceleration and deceleration (both apply on ground and in air), there is a possibility that players perceive the game feel differently on ground and in air. Instead of trying to describe both at the same time, players are shown two input fields.

After describing the game feel with the players' own words, a new set of questions are shown. Here, players are asked to rate the game feel on a 7-point Likert scale. Taking inspiration from Swink's book \cite{swink}, players rate the game feeling on how `twitchy', `fluid', `stiff', `floaty' and `responsive' they felt the controls were. Lastly, players are asked about how \textit{enjoyable}, \textit{difficult} and \textit{frustrating} it was to control the ball, as well as \textit{how much they liked} the control of the ball. In case players forgot how it felt to control the ball, they can always click on the \textit{Resume Playing} button and refresh their memories.

\begin{figure*}[htbp]
\centering
\includegraphics[width=0.95\textwidth]{Pics/game_phases}
\caption{When the player has collected three stars, the game pauses and shows a questionnaire.}
\label{fig:questionnaire}
\end{figure*}

\subsection{Gathering Data}
The game was uploaded to a server and shared on social media websites and gaming forums over the span of one week. A landing page was created where participants could choose to either play the game in their browsers or download a standalone build\footnote{The game is available here: \\ \url{http://tunnelvisiongames.com/g/GameFeel.html}.}. Additionally, Bitly \cite{bitly} was used to keep track of where the link was shared.

Whenever a player completed a round (collecting three stars and answering the questionnaire), data was  sent to an MySQL database. The data entry includes demographical information (age, gender, region, previous experience with games), parameter information (acceleration and deceleration times) and player descriptions (how players describe the game feel, and how they rate the game feel on the pre-defined scales). Target platform (web player or Windows player), player death count, average framerate and time spent on the level were also stored.

To ensure the order in the Latin square (see Section \ref{latinSection}), players were assigned a number between 1 and 4 when starting the game. This number corresponds to the sequences in Table \ref{table:latin}. This was achieved by taking modulus 4 of the total amount of players having completed the test and adding one to it. For instance, if 26 participants had played the game before entering, the next player would be assigned the sequence number (26 \% 4) + 1 = 3.
\section{Experimental Design} \label{experimentalDesign}
%\subsection{Research Question}
%How does the acceleration and deceleration of a player avatar influence the game feel of a 2D platformer? What types of words do players use to describe the game feeling, and what parameters result in these words?
%\subsection{Stimuli}
%The time it took for the player avatar to accelerate and decelerate changed between rounds.

\subsection{Stimulus Presentation}
Participants played four rounds of the game. Each round had different acceleration and deceleration times. All other factors were held constant, e.g., the level design, sound effects and the parameters for jumping. Between rounds, participants were asked to describe how it felt to play the game. The term \textit{game feel} was explicitly not explained, so that participants would try to describe the feel of the game from their own understanding of what game feel might be.

The experiment was designed as a repeated-measures, within-participant design \cite{cunningham}. There are a total of four possible combinations, as is shown in Table \ref{tab:combinations}. One of the strengths with within-participant designs is that it doesn't require as many participants as a between-participant design, since each participant will try all the conditions. However, one disadvantage is the risk of carry-over effects \cite{experimental1}. This might be due to fatigue (e.g., participants becoming bored after having experienced the multiple conditions) or practice (e.g., participants are better at the end than when they started). To balance potential ordering effects, a Latin square design was used (see Table \ref{table:latin}).

\begin{table} \centering
\caption{Four different combinations.}
\label{tab:combinations}
\begin{tabular}{ccc}
\toprule
& \textbf{Acceleration} & \textbf{Deceleration} \\
\midrule
\textbf{Stimulus 1} & Fast (A) & Fast (A)\\
\textbf{Stimulus 2} & Slow (B) & Slow (B)\\
\textbf{Stimulus 3} & Fast (A) & Slow (B)\\
\textbf{Stimulus 4} & Slow (B) & Fast (A)\\
\bottomrule
\end{tabular}
\end{table}

%The order in which the different stimuli are shown can affect the participants' behaviour/perception. Therefore, a Latin square design was used in order to balance these potential ordering effects 
%A way to prevent this is to use a counter-balanced design. This method reduces the risks of the order influencing the results \cite{experimental2}. Ideally, since there are four possible conditions, there should be 4x3x2x1 different orders, i.e., 24 orders of treatment. The number of participants must also be a multiple of 24, since there should be an equal number in each group \cite{experimental2}. Having 24 different combinations was deemed too complex; thus, an incomplete balanced design in the form of Latin squares was used instead (see Table \ref{table:latin}). Even though the order effects aren't eliminated completely, they become balanced.

\begin{table} \centering
\scriptsize
\caption{Latin squares are arranged in rows and columns such that each of the stimuli conditions only occur once in each row and column. The first letter is the acceleration, the second letter is the deceleration. `A' means fast and `B' means slow.}
\label{table:latin}
\begin{tabular}{ccccc}
\toprule
& \textbf{Stimulus 1} & \textbf{Stimulus 2} & \textbf{Stimulus 3} & \textbf{Stimulus 4}\\
\midrule
\textbf{Seq. 1} & AA & BB & BA & AB\\
\textbf{Seq. 2} & BB & AB & AA & BA\\
\textbf{Seq. 3} & AB & BA & BB & AA\\
\textbf{Seq. 4} & BA & AA & AB & BB\\
\bottomrule
\end{tabular}
\end{table}

\subsection{Task} \label{task}
Players are tasked with collecting three stars in each round. The purpose of the stars is to ensure that players move/jump around enough in order to experience the feel of controlling the ball. The stars have been placed in the beginning, middle and end of the level, so players have to experience the whole level each time. The level consists of traditional platforming elements, as well as obstacles in the form of a few moving enemies and spikes. Figure \ref{fig:level} shows an overview of the game's level.

\begin{figure*}[htbp]
\centering
\includegraphics[width=1\textwidth]{Pics/levelStructure}
\caption{Participants played the same level four times.}
\label{fig:level}
\end{figure*}

In order to gather responses from the test participants, a questionnaire was built into the game. Each time players collect three stars, the game is paused and a questionnaire is shown (see Figure \ref{fig:questionnaire}). The questionnaire consists of three sets of questions. The first asks players to try and describe the feeling of controlling the ball on the ground and in the air, with their own words. It is stressed that the chosen word(s) should be something that the player would use to describe the feeling to a friend. There were no restrictions on what types of words players could use (the input field's length is equivalent to approximately 66 characters, but it is possible to scroll forward/backward if a participant writes more than this). Each input field includes five randomly-chosen example words to give participants an idea of what could be used (see Table \ref{table:wordsExamples}).
%Participants were asked to describe the feeling both on the ground and in air. Even though only horizontal movement is changed in the form of the acceleration and deceleration (both apply on ground and in air), there is a possibility that players perceived the game feel differently on ground and in air. Instead of trying to describe both at the same time, players were shown two input fields.

\begin{table} \centering
\scriptsize
\caption{Participants were shown five randomly-chosen words to give them an idea of what they could write.}
\label{table:wordsExamples}
\begin{tabular}{ccccc}
\toprule
Fragile & Rigid & Firm & Solid & Thick\\
Fixed & Robust & Sore & Steadfast & Wild\\
Constant & Free & Hard & Tough & Restricted\\
Limited & Reduced & Fast & Heavy & Slow\\
Enjoyable & Stressful & Annoying & Realistic & Normal\\
Difficult & Easy & Dry & Juicy & Mechanical\\
Automatic & Organic & Exciting & Wet & Simple\\
Complicated & Direct & Inert & Unrealistic & Light\\
\bottomrule
\end{tabular}
\end{table}

After describing the game feel with the players' own words, a new set of questions was shown. Here, players were asked to rate the game feel on a 7-point Likert scale. Taking inspiration from Swink \cite{swink}, players rated the game feeling on how \textit{twitchy}, \textit{fluid}, \textit{stiff}, \textit{floaty} and \textit{responsive} they felt the controls were. Lastly, players were asked about how \textit{enjoyable}, \textit{difficult} and \textit{frustrating} it was to control the ball, as well as \textit{how much they liked} the control of the ball. In case players forgot how it felt to control the ball, they could always click on the \textit{Resume Playing} button to refresh their memories before continuing with the questionnaire.

After finishing the fourth round, participants were asked to complete an online post-questionnaire. The questions here were about game feel in general.

\begin{figure*}[htbp]
\centering
\includegraphics[width=0.95\textwidth]{Pics/game_phases}
\caption{When the player collected three stars, the game paused and showed a questionnaire.}
\label{fig:questionnaire}
\end{figure*}

\subsection{Participants}
The game was uploaded to a server and shared on social media websites and gaming forums. The primary target group was people who already play videogames; however, others were welcome to play the game as well. The participants were oblivious to the exact purpose and methods used in the experiment; they only knew that the research topic was about game feel (however, the term was never explained), but not how this was measured.

A web page was created where participants could choose to either play the game in their browser or download a standalone build. Whenever a player completed a round (by collecting three stars and answering the questionnaire), data was sent to an MySQL database.

%\footnote{The game is available here: \\ \url{http://tunnelvisiongames.com/g/GameFeel.html}}


%To ensure the order in the Latin square (see Section \ref{latinSection}), players were assigned a number between 1 and 4 when starting the game. This number corresponds to the sequences in Table \ref{table:latin}. This was achieved by taking modulus 4 of the total amount of participants having completed the experiment and adding 1 to it. For instance, if 26 participants had played the game before entering, the next player would be assigned the sequence number (26 \% 4) + 1 = 3.
\section{Data Analysis} \label{data}
At the time of writing, 274 participants have played the game.

\subsection{Demographics}
As stated previously, the game was mainly shared on gaming websites. Tables \ref{table:demographics1} and \ref{table:demographics2} show demographical data about the participants. Most of the participants rated themselves rather experienced with both playing videogames in general and playing 2D platforming games.


%%% This is how to insert a table %%%
\begin{table}[htbp]
\scriptsize
\centering
\begin{tabular}{|l|l|l|l|}
\hline
\textbf{Platform} & Windows Web & Windows Exe & Mac Web \\
                  & 55,2\%      & 30,7\%      & 14,1\%  \\
\hline
\textbf{Gender}   & Male        & Female      & Other   \\
                  & 93,6\%      & 5,7\%       & 0,7\%   \\
\hline
\textbf{Averages}      & Age     & Framerate            & Time spent        \\
                  & 24,6 years  & 59,7 FPS           & 61 seconds/level        \\
\hline
\end{tabular}
\caption{Demographics data 1.}
\label{table:demographics1}
\end{table}


tex text text


\begin{table}[h]
\scriptsize
\centering
\begin{tabular}{|l|ll|}
\hline
\textbf{Region}                      &                & \textbf{}               \\ \hline
\textit{Europe}                      & 70,6\%         &                         \\ \hline
\textit{Americas}                    & 26,7\%         &                         \\ \hline
\textit{Asia}                        & 1,9\%          &                         \\ \hline
\textit{Oceania}                     & 0\%            &                         \\ \hline
\textit{Africa}                      & 0,6\%          &                         \\ \hline
\textit{Other}                       & 0,2\%          &                         \\ \hline
\textit{\textbf{Experience with...}} & \textbf{Videogames} & \textbf{2D platformers} \\ \hline
\textit{1 (none)}                           & 0\%            & 0\%                     \\ \hline
\textit{2}                           & 0\%            & 3,1\%                   \\ \hline
\textit{3}                           & 0,6\%          & 10,1\%                  \\ \hline
\textit{4}                           & 4\%            & 14,1\%                  \\ \hline
\textit{5}                           & 9,6\%          & 23,2\%                  \\ \hline
\textit{6}                           & 22,5\%         & 16,9\%                  \\ \hline
\textit{7 (a lot)}                           & 63,3\%         & 32,6\%                  \\ \hline
\end{tabular}
\caption{Demographics data 2.}
\label{table:demographics2}
\end{table}


As stated in Section \ref{latinSection}, ideally there should be an equal amount of participants in each of the four Latin square sequences. However, as shown in Table \ref{table:latinSequenceNumber}, this is not the case. This might be due to either players quitting halfway, or that participants for some reason lost the Internet connection while trying to access the current Latin square number (in this case, the sequence was chosen randomly). The average number of completed levels were 2,5 per participant. This might be due to the fatigue mentioned in Section \ref{latinSection}. 274 have begun the game, i.e., playing the first level after the two anchoring examples mentioned in Section \ref{task}. Each time a participant collects three stars and answer the questionnaire, data is logged. In total, there were 701 data logs. Figure \ref{fig:retention} illustrates the retention rate of the participants.

%%% This is how to insert a table %%%
\begin{table}[htbp]
\scriptsize
\centering
\begin{tabular}{|c|c|}
\hline
 & \textbf{\# of participants}\\
 \hline
\textbf{Sequence 1} & 190\\
\hline
\textbf{Sequence 2} & 179\\
\hline
\textbf{Sequence 3} & 165\\
\hline
\textbf{Sequence 4} & 167\\
\hline
\end{tabular}
\caption{The number of participants in each of the four Latin square sequences.}
\label{table:latinSequenceNumber}
\end{table}

\begin{figure}[htbp]
\centering
\includegraphics[width=0.4\textwidth]{Pics/retetionRate}
\caption{Due fatigue or other factors, not all participants played the four rounds.}
\label{fig:retention}
\end{figure}



\section{Discussion \& Conclusion} \label{discussion}
When it comes to how people describe game feel in their own words, it seems that many use basic descriptions such as `heavy', `slow', `responsive', `realistic', `sluggish' and `floaty'. For this type of game those description might be enough to describe the game feel.

There are a few correlations between the acceleration, deceleration and the pre-defined words. However, it seems like there are different interpretations of what those words mean, e.g., what it means for a game to feel \textit{stiff}.

While some participants seemed quite sensitive to even smaller changes, others reported that they didn't feel any difference between the rounds.

The curves presented in this paper are based on averages. The results are less clear-cut when compared to those provided by Swink (see Section \ref{design}). In this experiment, participants were asked to rate the acceleration and deceleration values along a group of Likert scales. An alternative approach would be to use A/B testing and ask participants whether stimulus A felt more or less \textit{twitchy} than stimulus B. It would also be interesting to let players themselves tweak and tune the game's parameters in order to achieve what they think feels best.

Furthermore, one could look into more evenly-distributed sequences of the stimuli, say, within the 240 ms interval. Also, it might be interesting to look into non-linear curves and the influence of how responsive this feels (e.g., a slow acceleration that starts with an initial bump in its curve).

It appears that some participants were confused by the term \textit{game feel}. This is to be expected, since it's still a loose and relatively unknown concept about how players perceive games. A similar term, \textit{mouthfeel}, describes, as the name suggests, the sensation of food in the mouth. Even though the term was coined in 1951 \cite{mouthfeel}, it is still a relatively unknown concept. It might take a while before game feel finds its place in the vocabulary of the common game player.

Game feel is a holistic experience with many contributing factors. In this experiment, only the avatar movement was looked at. However, all other elements still indirectly influence the overall game feel, such as the rolling animation of the ball; the gravity and jumping mechanic; the level design; and the input device. It is unclear how big an influence the altering acceleration/deceleration has over these other factors, as well as other contributing elements such as the attention, mood and motivation of the player. Further research into the influence of these is required.

\bibliographystyle{abbrv}
%\begin{verbatim}

%\end{verbatim}
\bibliography{references}  % sigproc.bib is the name of the Bibliography in this case

\balancecolumns
% That's all folks!
\end{document}
